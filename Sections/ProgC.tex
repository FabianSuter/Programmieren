\section{ProgC}

    \subsection{Wichtige Kurzbefehle}
		\begin{tabular}{ll}
			\verb|cd "Path"| & Pfad anwählen \\
			\verb|cd ..| & um eine Ebene nach oben (zurück) \\
			\verb|mkdir "Ordnername"| & Ordner erstellen \\
			\verb|rmkdir "Ordnername"| & Ordner löschen \\
			\verb|rm -rf *| & Alles innerhalb vom aktuellen Ordner löschen \\
			\verb|rm "Datei"| & Datei löschen \\
			\verb|mv "Name alt" "Name neu"| & Datei umbenennen \\
			\verb|cp "Datei alt" "Datei neu"| & Datei kopieren und benennen \\
			\verb|clang -Wall -o "Dateiname" "Dateiname.c"| & clang-Compiler mit Warnungen \\
			\verb|clang -Wall -o "Dateiname" "Dateiname.c" -lm| & -lm für Mathebibliothek \\
			\verb|ls| & Listet alle Files im akt. Verzeichnis auf \\
			\verb|ls -l| & Inkl. Informationen wie Grösse u.a. \\
			\verb|ls -a| & Inkl. versteckten Dateien \\
			\verb|ls -al| & Beide Varianten \\
		\end{tabular}

	\subsection{Zahlensysteme}
		\begin{tabular}{|l|l|l|l|l|l|l|l|}
			\hline
			$2^0$ = 1 & $2^1$ = 2 & $2^2$ = 4 & $2^3$ = 8 & $2^4$ = 16 & $2^5$ = 32 & $2^6$ = 64 & $2^7$ = 128 \\
			\hline
		\end{tabular}

		\begin{tabular}{|l|llllll|}
			\hline
			Oktal & 3 Bits & $X_8$ & $X_O$ & $X_q$ & $X_oct$ & $0X$ \\
			\hline
			Hex & 4 Bits   & $X_16$ & $X_h$ & $XH$ & $X_hex$ & 0xX \\
			\hline
		\end{tabular}

		\paragraph{Hexadezimal}
			\begin{tabular}{llllllllllllllll}
				0 & 1 & 2 & 3 & 4 & 5 & 6 & 7 & 8 & 9 & 10 & 11 & 12 & 13 & 14 & 15 \\
				0 & 1 & 2 & 3 & 4 & 5 & 6 & 7 & 8 & 9 & A & B & C & D & E & F \\
			\end{tabular}
			
		%Todo: ASCI hinzufügen
	
	\subsection{Datentypen und Variablen}
		

	\subsection{Code-Snippets}
		\subsubsection{Array und Pointer}
			\begin{lstlisting}[language=C]
#include <stdio.h>

int main(){
	enum{array_size = 6};
	int test[array_size] = {1,2,3,4,5,6};
	for(int i =0; i<array_size; ++i)
		printf("Element %u: %i\n", i, test[i]);
	
	printf("Groesster: %d", *findAbsMax(test, array_size));
	return 0;
}
			\end{lstlisting}
			Main-Funktion zum Finden eines \textbf{betragsmässig} grössten Wertes innerhalb eines Arrays.

			\begin{lstlisting}[language=C]
int* findAbsMax(int* arr, size_t size){
	int* max_ptr = &arr[0];
	for(size_t i = 0; i < size; ++i){
		if((arr[i] >=0 && *max_ptr >=0 && arr[i] > *max_ptr)
		|| (arr[i] <=0 && *max_ptr <=0 && arr[i] < *max_ptr)
		|| (arr[i] >=0 && *max_ptr <=0 && arr[i] > *max_ptr * -1)
		|| (arr[i] <=0 && *max_ptr >=0 && arr[i] * -1 > *max_ptr))
			max_ptr = &arr[i];
	}
	return max_ptr;
}
			\end{lstlisting}